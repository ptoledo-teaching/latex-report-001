%%
%% PT Report Template
%% Full-featured template demonstrating all capabilities
%%
%% Author: Pedro Toledo Correa
%% Version: 0.1
%% Date: 2025-10-18
%%

\documentclass{pt-report}
% Available options:
% - Language: spanish (default), english, portuguese, french
% - Code: nominted (disable minted, use verbatim)
% - Font size: 10pt, 11pt, 12pt (default)

% Document metadata
\title{Informe Técnico del Proyecto}
\titlesub{Sistema de Gestión Académica}
\titlesubsub{Informe Final}

% Version control
\version{1.0}
\build{auto}  % Use 'auto' for automatic build counting or specify a number
% \watermark{BORRADOR}  % Uncomment to add watermark

% Authors (add as many as needed)
\addauthor{Pedro}{Toledo}{pedro.toledo@universidad.cl}{}

% Optional: Add institutional logo
% \logo{figures/logo.png}

% Academic/Class information (optional)
\classcode{IWI-131}
\classname{Programación}
\classsemester{Primer Semestre 2025}

% Institution information (optional)
\department{Departamento de Informática}
\school{Escuela de Ingeniería}
\university{Universidad Técnica Federico Santa María}

\begin{document}

\maketitle

% Table of contents
\tableofcontents
\ptlistoffigures  % Only shown if figures exist
\ptlistoftables   % Only shown if tables exist
\ptlistofcodes    % Only shown if code listings exist

\section{Introducción}

Esta es una plantilla completa que demuestra todas las capacidades de la clase
\inlinecode{pt-report}. El documento está configurado en formato de una columna
con márgenes de 1 pulgada y cada sección comienza en una nueva página.

\subsection{Características Principales}

La clase \inlinecode{pt-report} extiende la clase estándar \inlinecode{article}
de \LaTeX{} con las siguientes características:

\begin{itemize}
    \item Diseño profesional de una columna con márgenes cómodos
    \item Generación automática de página de título
    \item Gestión avanzada de autores con afiliaciones
    \item Control de versiones y numeración de compilaciones
    \item Saltos de página automáticos entre secciones
    \item Formato personalizado de secciones
    \item Cajas de resaltado para contenido importante
    \item Todas las características del paquete \inlinecode{pt-commons}
\end{itemize}

\subsection{Opciones de la Clase}

El documento puede configurarse con diferentes opciones:

\begin{enumerate}
    \item \textbf{Idiomas:} spanish, english, portuguese, french
    \item \textbf{Código:} nominted (desactiva minted)
    \item \textbf{Tamaño de fuente:} 10pt, 11pt, 12pt (por defecto)
\end{enumerate}

\subsection{Estructura del Informe}

Este informe está organizado de la siguiente manera:

\begin{description}
    \item[Sección 1:] Introducción y contexto del proyecto
    \item[Sección 2:] Marco teórico y conceptos fundamentales
    \item[Sección 3:] Metodología empleada en el desarrollo
    \item[Sección 4:] Resultados obtenidos y análisis
    \item[Sección 5:] Conclusiones y trabajo futuro
\end{description}

\section{Marco Teórico}

\subsection{Fundamentos Conceptuales}

El marco teórico establece las bases conceptuales sobre las cuales se desarrolla
el proyecto. En esta sección se presentan los conceptos fundamentales necesarios
para comprender el trabajo realizado.

\begin{highlightbox}
    \textbf{Concepto Importante:} La arquitectura del sistema se basa en el patrón
    Model-View-Controller (MVC), que separa la lógica de negocio de la presentación
    y permite un desarrollo más modular y mantenible.
\end{highlightbox}

\subsection{Estado del Arte}

El estado del arte presenta una revisión de los trabajos previos relacionados
con el tema del informe.

\subsubsection{Trabajos Relacionados}

Diversos autores han abordado problemas similares:

\begin{itemize}
    \item Smith et al. (2023) presentaron un sistema de gestión basado en microservicios
    \item González y Pérez (2024) propusieron una arquitectura escalable
    \item Martínez (2024) desarrolló un framework de pruebas automatizadas
\end{itemize}

\subsection{Herramientas y Tecnologías}

Las principales herramientas y tecnologías utilizadas se resumen en la siguiente tabla:

\begin{center}
    \begin{tblr}{colspec={|l|l|l|}}
        \hline
        \tableheader
        Categoría            & Herramienta & Versión \\
        \hline
        \hline
        Backend              & Python      & 3.11    \\
        \hline
        Frontend             & React       & 18.2    \\
        \hline
        Base de Datos        & PostgreSQL  & 15.0    \\
        \hline
        Control de Versiones & Git         & 2.40    \\
        \hline
        Contenedores         & Docker      & 24.0    \\
        \hline
    \end{tblr}
\end{center}

\section{Metodología}

\subsection{Enfoque Metodológico}

La metodología empleada consta de las siguientes etapas principales:

\begin{enumerate}
    \item \textbf{Análisis de Requisitos:} Identificación y documentación de los requisitos funcionales y no funcionales del sistema.
    \item \textbf{Diseño:} Creación de la arquitectura del sistema y diseño de componentes individuales.
    \item \textbf{Implementación:} Desarrollo del código fuente siguiendo las mejores prácticas.
    \item \textbf{Pruebas:} Verificación y validación del sistema mediante pruebas unitarias, de integración y de sistema.
    \item \textbf{Despliegue:} Puesta en producción del sistema en el ambiente objetivo.
\end{enumerate}

\subsection{Análisis de Requisitos}

Los requisitos se clasificaron en dos categorías principales:

\paragraph{Requisitos Funcionales}

Los requisitos funcionales especifican lo que el sistema debe hacer:

\begin{itemize}
    \item RF-01: El sistema debe permitir el registro de usuarios
    \item RF-02: El sistema debe gestionar cursos y horarios
    \item RF-03: El sistema debe generar reportes personalizados
    \item RF-04: El sistema debe enviar notificaciones por email
\end{itemize}

\paragraph{Requisitos No Funcionales}

Los requisitos no funcionales especifican cómo debe comportarse el sistema:

\begin{itemize}
    \item RNF-01: El sistema debe responder en menos de 2 segundos
    \item RNF-02: El sistema debe soportar 1000 usuarios concurrentes
    \item RNF-03: El sistema debe tener 99.9\% de disponibilidad
    \item RNF-04: El sistema debe cumplir con la ley de protección de datos
\end{itemize}

\subsection{Diseño del Sistema}

El diseño del sistema se basa en una arquitectura de tres capas:

\begin{description}
    \item[Capa de Presentación:] Interfaz de usuario desarrollada en React
    \item[Capa de Lógica de Negocio:] API REST desarrollada en Python con Flask
    \item[Capa de Datos:] Base de datos PostgreSQL con esquema normalizado
\end{description}

La comunicación entre capas se realiza mediante API REST con formato JSON.

\subsection{Implementación}

La implementación siguió un enfoque iterativo e incremental. A continuación
se muestra un ejemplo del código principal del servidor:

\begin{ptprintcode}{python}
    from flask import Flask, jsonify, request
    from flask_cors import CORS
    from database import db
    from models import User, Course

    app = Flask(__name__)
    CORS(app)
    app.config['SQLALCHEMY_DATABASE_URI'] = 'postgresql://localhost/academic'
    db.init_app(app)

    @app.route('/api/users', methods=['GET'])
    def get_users():
    """Obtiene la lista de todos los usuarios"""
    users = User.query.all()
    return jsonify([user.to_dict() for user in users])

    @app.route('/api/courses', methods=['GET'])
    def get_courses():
    """Obtiene la lista de todos los cursos"""
    courses = Course.query.all()
    return jsonify([course.to_dict() for course in courses])

    if __name__ == '__main__':
    app.run(debug=True, port=5000)
\end{ptprintcode}

\subsection{Pruebas}

Se implementaron tres niveles de pruebas:

\begin{enumerate}
    \item \textbf{Pruebas Unitarias:} Verifican el funcionamiento de componentes individuales
    \item \textbf{Pruebas de Integración:} Verifican la interacción entre componentes
    \item \textbf{Pruebas de Sistema:} Verifican el funcionamiento completo del sistema
\end{enumerate}

La cobertura de código alcanzada fue del 87\%, superando el objetivo del 80\%.

\section{Resultados}

\subsection{Resultados Cuantitativos}

Los resultados cuantitativos del proyecto se resumen en la siguiente tabla:

\begin{center}
    \begin{tblr}{colspec={|l|c|c|c|}}
        \hline
        \tableheader
        Métrica                  & Objetivo & Obtenido & Cumplimiento                        \\
        \hline
        \hline
        Tiempo de respuesta (ms) & $< 2000$ & 1200     & \textcolor{ptgreen}{\faIcon{check}} \\
        \hline
        Usuarios concurrentes    & $> 1000$ & 1500     & \textcolor{ptgreen}{\faIcon{check}} \\
        \hline
        Disponibilidad (\%)      & $> 99.9$ & 99.95    & \textcolor{ptgreen}{\faIcon{check}} \\
        \hline
        Cobertura de código (\%) & $> 80$   & 87       & \textcolor{ptgreen}{\faIcon{check}} \\
        \hline
        Líneas de código         & N/A      & 12,500   & N/A                                 \\
        \hline
    \end{tblr}
\end{center}

Todos los objetivos cuantitativos fueron cumplidos satisfactoriamente.

\subsection{Resultados Cualitativos}

Los resultados cualitativos incluyen:

\begin{itemize}
    \item Interfaz de usuario intuitiva y fácil de usar
    \item Código fuente bien documentado y mantenible
    \item Arquitectura escalable que permite crecimiento futuro
    \item Buena aceptación por parte de los usuarios finales
\end{itemize}

\subsection{Análisis de Resultados}

El análisis de los resultados obtenidos muestra que:

\begin{highlightbox}
    \textbf{Hallazgo Principal:} El sistema desarrollado cumple con todos los
    requisitos establecidos y supera las expectativas en términos de rendimiento
    y escalabilidad. La arquitectura basada en microservicios ha demostrado ser
    una elección acertada para el proyecto.
\end{highlightbox}

La comparación con sistemas similares muestra que nuestra solución ofrece
mejor rendimiento en un 35\% y mayor escalabilidad.

\subsection{Fórmulas y Ecuaciones}

El cálculo del tiempo de respuesta promedio se realizó mediante:

\begin{equation}
    \overline{t} = \frac{1}{n} \sum_{i=1}^{n} t_i
    \label{eq:tiempo-promedio}
\end{equation}

donde $t_i$ representa el tiempo de respuesta de la solicitud $i$ y $n$ es el
número total de solicitudes.

La desviación estándar del tiempo de respuesta es:

\begin{equation}
    \sigma = \sqrt{\frac{1}{n-1} \sum_{i=1}^{n} (t_i - \overline{t})^2}
    \label{eq:desviacion}
\end{equation}

Los resultados muestran una desviación estándar baja ($\sigma = 150$ ms),
indicando consistencia en los tiempos de respuesta.

\section{Conclusiones}

\subsection{Conclusiones Generales}

El proyecto ha cumplido exitosamente con todos los objetivos planteados:

\begin{enumerate}
    \item Se desarrolló un sistema completo y funcional de gestión académica
    \item Se implementaron todas las funcionalidades requeridas
    \item Se superaron las métricas de rendimiento establecidas
    \item Se documentó exhaustivamente el proceso de desarrollo
\end{enumerate}

\subsection{Lecciones Aprendidas}

Durante el desarrollo del proyecto se aprendieron varias lecciones importantes:

\begin{itemize}
    \item La planificación detallada al inicio ahorra tiempo en etapas posteriores
    \item Las pruebas automatizadas son esenciales para mantener la calidad
    \item La comunicación efectiva entre el equipo es fundamental
    \item La documentación debe realizarse de forma continua, no al final
\end{itemize}

\subsection{Trabajo Futuro}

Las siguientes mejoras se proponen para versiones futuras:

\begin{enumerate}
    \item Implementar autenticación mediante OAuth2
    \item Agregar soporte para aplicaciones móviles nativas
    \item Mejorar el sistema de notificaciones con push notifications
    \item Implementar analíticas avanzadas con machine learning
    \item Agregar soporte multiidioma en la interfaz
\end{enumerate}

\subsection{Reflexión Final}

Este proyecto ha demostrado que es posible desarrollar sistemas complejos
de alta calidad siguiendo metodologías ágiles y buenas prácticas de ingeniería
de software. Los resultados obtenidos superan las expectativas iniciales y
establecen una base sólida para futuros desarrollos.

\begin{highlightbox}
    \textbf{Reflexión:} El éxito del proyecto se debe principalmente al trabajo
    en equipo, la planificación adecuada y el uso de tecnologías modernas y
    probadas. La experiencia adquirida será invaluable para proyectos futuros.
\end{highlightbox}

\end{document}
