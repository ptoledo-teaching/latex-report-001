
% Packages
\usepackage[utf8]{inputenc}
\usepackage[spanish]{babel}
\usepackage{multirow}
\usepackage{graphicx}
\usepackage{listings}
\usepackage{longtable}
\usepackage[hidelinks]{hyperref}
\usepackage{geometry}
\usepackage[sfdefault,lf]{carlito}
\usepackage[T1]{fontenc}
\usepackage{fancyhdr}
\usepackage[table]{xcolor}
\usepackage{totcount}
\usepackage{float}
\usepackage{datetime}

% Packages configurations
\regtotcounter{table}
\regtotcounter{figure}
\geometry{
  letterpaper,
  %showframe,
  top=1.0in,
  bottom=1.0in,
  left=1.25in,
  right=1.25in,
  headheight=15pt,
}
\pagestyle{fancy}
\definecolor{lightgray}{rgb}{.9,.9,.9}
\definecolor{darkgray}{rgb}{.4,.4,.4}
\definecolor{purple}{rgb}{0.65, 0.12, 0.82}
\lstdefinelanguage{JavaScript}{
  keywords={typeof, new, true, false, catch, function, return, null, catch, switch, var, if, in, while, do, else, case, break},
  keywordstyle=\color{blue}\bfseries,
  ndkeywords={class, export, boolean, throw, implements, import, this},
  ndkeywordstyle=\color{darkgray}\bfseries,
  identifierstyle=\color{black},
  sensitive=false,
  comment=[l]{//},
  morecomment=[s]{/*}{*/},
  commentstyle=\color{purple}\ttfamily,
  stringstyle=\color{red}\ttfamily,
  morestring=[b]',
  morestring=[b]"
}
\lstset{
  language=JavaScript,
  backgroundcolor=\color{lightgray},
  extendedchars=true,
  basicstyle=\footnotesize\ttfamily,
  showstringspaces=false,
  showspaces=false,
  numbers=left,
  numberstyle=\footnotesize,
  numbersep=9pt,
  tabsize=2,
  breaklines=true,
  showtabs=false,
  captionpos=b
}

% Document configurations
\hbadness=20000
\setlength\parskip{1em}
\setlength\parindent{15pt}
\renewcommand*\oldstylenums[1]{\carlitoOsF #1}
\renewcommand{\headrulewidth}{0.5pt}
\renewcommand{\footrulewidth}{0.5pt}
\cfoot{\thepage}
\graphicspath{{figures/}}

% Macros
\newcommand{\printif}[3]{\ifcsname @#1\endcsname#2\else#3\fi}

% Document beggining
\newif\ifcover
\covertrue
\AtBeginDocument{
  \ifcover
    \renewcommand\listtablename{Índice de tablas}
    \renewcommand\tablename{Tabla}
    \maketitle
  \fi
}

% Report definitions
\makeatletter
  \def\institution#1{\gdef\@institution{#1}}
  \def\classcode#1{\gdef\@classcode{#1}}
  \def\classname#1{\gdef\@classname{#1}}
  \def\classsemester#1{\gdef\@classsemester{#1}}
  \def\classparallel#1{\gdef\@classparallel{#1}}
  \def\doctitle#1{\gdef\@doctitle{#1}}
  \def\docsubtitle#1{\gdef\@docsubtitle{#1}}
  \def\version#1{\gdef\@version{#1}}
  \def\astudentname#1{\gdef\@astudentname{#1}}
  \def\astudentlastname#1{\gdef\@astudentlastname{#1}}
  \def\astudentrol#1{\gdef\@astudentrol{#1}}
  \def\astudentemail#1{\gdef\@astudentemail{#1}}
  \def\bstudentname#1{\gdef\@bstudentname{#1}}
  \def\bstudentlastname#1{\gdef\@bstudentlastname{#1}}
  \def\bstudentrol#1{\gdef\@bstudentrol{#1}}
  \def\bstudentemail#1{\gdef\@bstudentemail{#1}}
  \def\cstudentname#1{\gdef\@cstudentname{#1}}
  \def\cstudentlastname#1{\gdef\@cstudentlastname{#1}}
  \def\cstudentrol#1{\gdef\@cstudentrol{#1}}
  \def\cstudentemail#1{\gdef\@cstudentemail{#1}}
\makeatother
  % Setting fancy headers
\makeatletter
  \fancyhead[LE]{\printif{classcode}{\@classcode\ - }{}\@classsemester}
  \fancyhead[RE]{\@doctitle}
  \fancyhead[LO]{\@astudentlastname\printif{bstudentlastname}{, \@bstudentlastname}{}\printif{cstudentlastname}{, \@cstudentlastname}{}}
  \fancyhead[RO]{\printif{version}{\@version\ - }{}\the\year/\two@digits{\month}/\two@digits{\day}}
\makeatother

% Redefinning titlepage
\makeatletter
\def\@maketitle{
  % Portada
  \thispagestyle{empty}
  \noindent\includegraphics[width=.475 \textwidth]{../latex-report-001/logo-\@institution}
  \vfill
  \vfill
  \begin{center}
    \printif{docsubtitle}{\fontsize{25}{25}\selectfont \@docsubtitle\\[3em]}{}
    {\fontsize{35}{35}\selectfont \@doctitle}\\[2.0em]
    {\fontsize{15}{15}\selectfont \printif{classcode}{\@classcode\ - }{}\@classsemester\printif{classparallel}{ - \@classparallel}{}}\\[10pt]
    {\fontsize{20}{20}\selectfont \@classname}\\[5pt]
    {\fontsize{15}{15}\selectfont \today\printif{version}{ - \@version}{}}
  \end{center}
  \vfill
  \vfill
  \vfill
  \begin{flushright}
    \begin{tabular}{r|l}
      \printif{astudentname}{\@astudentname\ \@astudentlastname & \@astudentrol \\ \@astudentemail & \\}{}
      \printif{bstudentname}{\@bstudentname\ \@bstudentlastname & \@bstudentrol \\ \@bstudentemail & \\}{}
      \printif{cstudentname}{\@cstudentname\ \@cstudentlastname & \@cstudentrol \\ \@cstudentemail & \\}{}
    \end{tabular}
  \end{flushright}
  \newpage
  % Índice
  \tableofcontents
  % Tablas
  \ifnum \totvalue{table}>0
    \listoftables
  \fi
  % Figuras
  \ifnum \totvalue{figure}>0
    \listoffigures
  \fi
  \newpage
}
\makeatother
